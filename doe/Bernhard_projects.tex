\documentclass[letterpaper,10pt]{article}


\setcounter{secnumdepth}{0}


\begin{document}




\section{Statistical hadronization models}

We perform an unbiased statistical analysis of hadron yields and ratios in heavy-ion collisions.  There are currently many conflicting statistical hadronization
models, and a variety of proposed best-fit parameters.  We seek to clarify this picture, particularly the controversial phase-space occupancy factors
($\gamma_q,\gamma_s$).

The statistical hadronization model of choice is \emph{Statistical Hadronization with Resonances} (SHARE), which uses a grand-canonical ensemble approach to
calculate hadron yields or ratios given a set of input parameters.  We go beyond standard $\chi^2$ minimization routines; notably, we employ a more flexible
Markov chain Monte Carlo (MCMC) approach.


\medskip\noindent [J.~E.~Bernhard, C.~E.~Coleman-Smith, and S.~A.~Bass]





\section{Onion visualization}

We visualize hadronic freeze-out surfaces as a function of strangeness in heavy-ion collisions.  It is well known that particles with more strangeness have
smaller cross sections, and therefore cease to interact earlier in the post-QGP expansion.  Perhaps the most fundamental set of freeze-out surfaces are those of
the baryons $\Omega$, $\Xi$, $\Lambda$, $p$.  When these surfaces are superimposed, the visual takes on a layered appearance, hence the name of the project
(``onion'').  

We extend the visualization to include time dependence.  The animation begins with an interacting QGP phase, from which the freeze-out surfaces of the various
baryons form as the medium expands and cools.


\medskip\noindent [J.~E.~Bernhard, H.~Petersen, and S.~A.~Bass]



\end{document}
