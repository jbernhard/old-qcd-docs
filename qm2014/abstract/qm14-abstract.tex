\documentclass[letterpaper,102pt]{article}


\pagestyle{empty}




\begin{document}

\noindent \textbf{QGP parameter extraction via a global analysis of event-by-event flow coefficient distributions} \\
Jonah E.\ Bernhard  and Steffen A.\ Bass (Duke University) \\[1ex]


\noindent A primary goal of heavy-ion physics is the measurement of the fundamental properties of the quark-gluon plasma (QGP), notably its
transport coefficients, such as the specific shear viscosity $\eta/s$.  Since these properties are not directly measurable, one relies on a
comparison of the data to computational models of the time-evolution of the collision to connect measured observables to the properties of
the transient QGP state.  The computational model parameters are tuned such that simulated observables optimally match experimental data.

Most studies to date are severely limited by computation time:  they typically rely on averaged quantities such as the average elliptic flow
coefficient $\langle v_2 \rangle$ for a given centrality bin---disregarding the effects of event-by-event fluctuations---and use ad hoc
methods for optimizing model parameters.  Often, each parameter is varied independently, while best-fit values are chosen via qualitative
comparisons to single observables.  This neglects correlations among parameters and leads to nebulous results lacking quantitative
uncertainty.

We propose a systematic model-to-data comparison method for extracting QGP properties.  First, a set of salient model parameters is chosen
for calibration---physical properties such as transport coefficients are of primary interest.  An event-by-event model is then evaluated at
many points in parameter space; this is made possible by recent advances in high-throughput computing.  Finally, a statistical surrogate
algorithm is used to interpolate the parameter space and determine the values which optimally reproduce experimental data.  This provides
rigorous constraints including quantitative uncertainty and sheds light on the relative importance of each parameter.  

The methodology is applied to a modern hybrid model with MC-Glauber and MC-KLN initial conditions, viscous 2+1D hydrodynamics, and the
hadron cascade UrQMD.  By leveraging the power of the Open Science Grid, we have run event-by-event simulations over wide ranges of several
crucial parameters, e.g.\ the shear viscosity and hydrodynamic thermalization time.  We calibrate the model to experimental event-by-event
flow distributions measured by the ATLAS experiment; these distributions are sensitive to initial-state fluctuations and therefore
constitute a more comprehensive probe of the QGP than event-averaged flow.

This massive-scale model-to-comparison yields new constraints on fundamental QGP properties and clarifies the essential features of a
physically accurate model.  The method is general and easily extensible to future studies.



\end{document}
