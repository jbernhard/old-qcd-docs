% abstract.tex

\documentclass[letterpaper,10pt]{article}


\pagestyle{empty}


\title{QGP parameter extraction via a global analysis of event-by-event flow coefficient distributions}
\author{Jonah Bernhard}


\begin{document}

\maketitle


\noindent Heavy-ion collisions produce a hot, dense phase of strongly-interacting matter known as the quark-gluon plasma (QGP) which quickly
($\sim 10^{-23}$ s) expands and freezes into hadrons.  The QGP cannot be observed directly---only final-state hadrons are detectable---so
computer models are used to indirectly characterize the medium.  A successful model must simulate realistic collision events and match its
final state to experimentally observed particles.

Modern models use a ``hybrid'' approach with relativistic hydrodynamics for the early hot and dense phase followed by non-equilibrium
Boltzmann transport for the dilute hadron-gas phase.  Hybrid models have provided approximate descriptions of a variety of observables, but are
very computationally expensive and therefore difficult to test systematically, so they remain poorly constrained.

One of the most important properties of the QGP is its shear viscosity to entropy density ratio $\eta/s$.  The QGP is postulated to behave
as a near-ideal fluid, hence its viscosity may be nearly minimal.  This is explored primarily via collective flow measurements---a natural
partner, since viscosity tends to damp collective behavior.  Previous studies have tentatively confirmed a small $\eta/s$, typically by
matching the centrality dependence of event-averaged flow coefficients between model and experiment.

The ATLAS experiment has recently measured event-by-event flow distributions, which could provide a much more sensitive probe
of $\eta/s$.  Using a hybrid model with MC-Glauber and MC-KLN initial conditions, viscous 2+1D hydrodynamics, and the hadron cascade UrQMD,
we calculate flow distributions over wide ranges of several model parameters including $\eta/s$.  By calibrating the model to data, we
extract the optimal values of each parameter and clarify the important features of a physically accurate model.


\end{document}
